\documentclass[aspectratio=169]{beamer}
\usetheme{Pittsburgh}
%Information to be included in the title page:
\title{\textsc{solid} principles}
\author{Ethan Kent}
\institute{Spoonflower}
\date{\today}
\usepackage{graphicx,microtype}
\usepackage[osf]{libertine}
% \usepackage{biblatex}
\usepackage[notes,backend=biber]{biblatex-chicago}
\addbibresource{solid.bib}

\begin{document}


\frame{\titlepage}

%%%%%%%%%%%%%%%%%%%%%%%%%%%%%%%%%%%%%%%%%%%%%%%%%%%%%%%%%%%%%%%%%%%%%%%%%%%%%%%%
% Main content
%%%%%%%%%%%%%%%%%%%%%%%%%%%%%%%%%%%%%%%%%%%%%%%%%%%%%%%%%%%%%%%%%%%%%%%%%%%%%%%%

\begin{frame}
  \frametitle{Purpose of the \textsc{solid} principles}
  \begin{quote}

    The \textsc{solid} principles tell us how to arrange our functions and data
    structures into~.~.~. grouping[s] of functions and data. The goal of the
    principles is the creation of mid-level software structures that:

    \begin{itemize}
      \item Tolerate change,
      \item Are easy to understand, and
      \item Are the basis of components that can be used in many software
            systems.\footnote{\cite[p.~58]{clean-arch}.}
    \end{itemize}

  \end{quote}
\end{frame}

\begin{frame}
  \frametitle{\textsc{solid} principles: not just for \textsc{oop}}
  \center
  \includegraphics[height=0.618\textheight]{bob-tweet}
  \footnote{\cite{martin-tweet}.}
\end{frame}

\begin{frame}
  \frametitle{The \textsc{solid} principles, listed}

  \begin{itemize}
    \item[S] The Single-Responsibility Principle.
    \item[O] The Open--Closed Principle.
    \item[L] The Liskov Substitution Principle.
    \item[I] The Interface-Segregation Principle.
    \item[D] The Dependency-Inversion Principle.
  \end{itemize}
\end{frame}

% ------------------------------------------------------------------------------
% SRP
% ------------------------------------------------------------------------------

\begin{frame}
  \frametitle{The Single-Responsibility Principle, defined}

  \begin{quote}
    A module should have one, and only one, reason to
    change.\footnote{\cite[p.~62]{clean-arch}.}
  \end{quote}

  \begin{quote}
    A module should be responsible to one, and only one, user or
    stakeholder.\footnote{\cite[p.~62]{clean-arch}.}
  \end{quote}

  \begin{quote}
    A module should be responsible to one, and only one,
    actor.\footnote{\cite[p.~62]{clean-arch}.}
  \end{quote}
\end{frame}

\begin{frame}
  \frametitle{The Single-Responsibility Principle, continued}

\end{frame}

% --------------------------------------

\begin{frame}
  \frametitle{The Single-Responsibility Principle, continued}

  The \emph{actor} or \emph{user/stakeholder} idea means code that accounting
  asks you to change shouldn't affect code that SEO asks you to change, for
  example.
\end{frame}

% ------------------------------------------------------------------------------
% Open–Closed
% ------------------------------------------------------------------------------

\begin{frame}
  \frametitle{The Open--Closed Principle, defined}

  \begin{quote}
    A software artifact should be open for extension but closed for
    modification.\footnote{\cite[p.~70]{clean-arch}.}
  \end{quote}
\end{frame}

% ------------------------------------------------------------------------------
% Liskov
% ------------------------------------------------------------------------------

\begin{frame}
  \frametitle{The Single-Responsibility Principle, defined}

  \begin{quote}
    Let $\phi(x)$ be a property provable about objects $x$ of type $T$. Then
    $\phi(y)$  should be true for objects $y$ of type $S$ where $S$
    is a subtype of $T$.\footnote{\cite[p.~1812]{liskov}.}
  \end{quote}
\end{frame}

% ------------------------------------------------------------------------------
% Interface Segregation
% ------------------------------------------------------------------------------

\begin{frame}
  \frametitle{The Single-Responsibility Principle, defined}

  \begin{quote}
    Keep interfaces small so that users don't end up depending on things they
    don't need.\footnote{\cite{solid-relevance}.}
  \end{quote}
\end{frame}

% ------------------------------------------------------------------------------
% Dependency Inversion
% ------------------------------------------------------------------------------

\begin{frame}
  \frametitle{The Single-Responsibility Principle, defined}

  \begin{quote}
    Depend in the direction of abstraction. High level modules should not depend
    upon low level details.\footnote{\cite{solid-relevance}.}
  \end{quote}
\end{frame}

%%%%%%%%%%%%%%%%%%%%%%%%%%%%%%%%%%%%%%%%%%%%%%%%%%%%%%%%%%%%%%%%%%%%%%%%%%%%%%%%
% Back matter
%%%%%%%%%%%%%%%%%%%%%%%%%%%%%%%%%%%%%%%%%%%%%%%%%%%%%%%%%%%%%%%%%%%%%%%%%%%%%%%%

\begin{frame}
  \frametitle{Bibliography}
  \begin{centering}
    \printbibliography
  \end{centering}
\end{frame}

\end{document}
